\documentclass{article}
% \usepackage[utf8]{inputenc} % Allows Unicode characters
\usepackage{amsmath}        % For math typesetting (optional)
\usepackage{graphicx}       % For including images (optional)

\title{Luby Sequence}
\author{I and Gen AIs}
\date{2025-06-20}

\begin{document}
\maketitle

The Luby sequence is a sequence of natural numbers defined recursively.
It is used in randomized algorithms and has applications in computer science.
Actually, I haven’t seen it used in any areas other than Boolean satisfiability.

The sequence is defined as follows:

\begin{equation}
Luby(k \ge 1) =
\begin{cases}
2^{i-1}, & \text{if } k = 2^i - 1 \text{ for some } i \geq 1, \\
Luby\left(k - 2^{i-1} + 1\right), & \text{if } 2^{i-1} \leq k < 2^i - 1
\end{cases}
\end{equation}


If we start the sequence from 0:

\begin{eqnarray*}
L(k \ge 0) &=& Luby(k+1) \\
&=&
\begin{cases}
2^{i-1}, & \text{if } k = 2^i -2 \text{ for some } i \geq 1, \\
Luby\left(k - 2^{i-1} + 2\right), & \text{if } 2^{i-1} \leq k + 1 < 2^i - 1
\end{cases}\\
&=&
\begin{cases}
2^{i-1}, & \text{if } k = 2^i -2 \text{ for some } i \geq 1, \\
L\left(k - 2^{i-1} + 1\right), & \text{if } 2^{i-1} \leq k + 1 < 2^i - 1
\end{cases}
\end{eqnarray*}

\end{document}
