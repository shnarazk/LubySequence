\documentclass{article}
% \usepackage[utf8]{inputenc} % Allows Unicode characters
\usepackage{amsmath}        % For math typesetting (optional)
\usepackage{graphicx}       % For including images (optional)

\title{Luby Sequence}
\author{shnarazk and Generative AIs}
\date{2025-06-20}

\begin{document}
\maketitle

The Luby sequence is a sequence of natural numbers defined recursively.
It is used in randomized algorithms and has applications in computer science.
However, outside of Boolean-satisfiability solvers, its applications
appear to be rare.

The sequence is defined as follows:

\begin{equation}
  Luby_1(k \ge 1) =
  \begin{cases}
    2^{i-1}, & \text{if } k = 2^i - 1 \text{ for some } i \geq 1, \\
    Luby_1\left(k - 2^{i-1} + 1\right), & \text{if } 2^{i-1} \leq k < 2^i - 1
  \end{cases}
\end{equation}

To obtain a zero-based version, define \(L(k)\) for \(k\ge0\) by
setting \(L(k)=L_1(k+1)\). Equivalently,

\begin{align}
  Luby(k \ge 0) &= Luby_1(k+1) \notag\\
  &=
  \begin{cases}
    2^{i-1}, & \text{if } k = 2^i -2 \text{ for some } i \geq 1, \\
    Luby_1\left(k - 2^{i-1} + 2\right), & \text{if } 2^{i-1} \leq k +
    1 < 2^i - 1
  \end{cases}\notag\\
  &=
  \begin{cases}
    2^{i-1}, & \text{if } k = 2^i -2 \text{ for some } i \geq 1, \\
    Luby\left(k - 2^{i-1} + 1\right), & \text{if } 2^{i-1} \leq k + 1 < 2^i - 1
  \end{cases}
\end{align}

For simplicity, we introduce a sequence $S_2$

\begin{align}
  S_2(k) = 2^{(k + 2).size}
\end{align}
where $size : N \longmapsto N $ returns the length of the binary
representation of the binary representation of its argument (as in Lean 4).

\begin{align}
  Luby(k)
  &=
  \begin{cases}
    S_2(k), & \text{if } S_2(k+2) = k+2, \\
    Luby\left(k + 1 - S_2(k)\right), & \text{otherwise}
  \end{cases}
\end{align}
\end{document}
